\documentclass[12pt,a4paper]{report}
\usepackage[utf8]{inputenc}
\usepackage[russian]{babel}
\usepackage[OT1]{fontenc}
\usepackage{amsmath}
\usepackage{amsfonts}
\usepackage{amssymb}
\usepackage{graphicx}
\usepackage{cmap}					% поиск в PDF
\usepackage{mathtext} 				% русские буквы в формулах
%\usepackage{tikz-uml}               % uml диаграммы

% TODOs
\usepackage[%
  colorinlistoftodos,
  shadow
]{todonotes}

% Генератор текста
\usepackage{blindtext}

%------------------------------------------------------------------------------

% Подсветка синтаксиса
\usepackage{color}
\usepackage{xcolor}
\usepackage{listings}
 
 % Цвета для кода
\definecolor{string}{HTML}{B40000} % цвет строк в коде
\definecolor{comment}{HTML}{008000} % цвет комментариев в коде
\definecolor{keyword}{HTML}{1A00FF} % цвет ключевых слов в коде
\definecolor{morecomment}{HTML}{8000FF} % цвет include и других элементов в коде
\definecolor{captiontext}{HTML}{FFFFFF} % цвет текста заголовка в коде
\definecolor{captionbk}{HTML}{999999} % цвет фона заголовка в коде
\definecolor{bk}{HTML}{FFFFFF} % цвет фона в коде
\definecolor{frame}{HTML}{999999} % цвет рамки в коде
\definecolor{brackets}{HTML}{B40000} % цвет скобок в коде
 
 % Настройки отображения кода
\lstset{
language=C, % Язык кода по умолчанию
morekeywords={*,...}, % если хотите добавить ключевые слова, то добавляйте
 % Цвета
keywordstyle=\color{keyword}\ttfamily\bfseries,
stringstyle=\color{string}\ttfamily,
commentstyle=\color{comment}\ttfamily\itshape,
morecomment=[l][\color{morecomment}]{\#}, 
 % Настройки отображения     
breaklines=true, % Перенос длинных строк
basicstyle=\ttfamily\footnotesize, % Шрифт для отображения кода
backgroundcolor=\color{bk}, % Цвет фона кода
%frame=lrb,xleftmargin=\fboxsep,xrightmargin=-\fboxsep, % Рамка, подогнанная к заголовку
frame=tblr
rulecolor=\color{frame}, % Цвет рамки
tabsize=3, % Размер табуляции в пробелах
showstringspaces=false,
 % Настройка отображения номеров строк. Если не нужно, то удалите весь блок
numbers=left, % Слева отображаются номера строк
stepnumber=1, % Каждую строку нумеровать
numbersep=5pt, % Отступ от кода 
numberstyle=\small\color{black}, % Стиль написания номеров строк
 % Для отображения русского языка
extendedchars=true,
literate={Ö}{{\"O}}1
  {Ä}{{\"A}}1
  {Ü}{{\"U}}1
  {ß}{{\ss}}1
  {ü}{{\"u}}1
  {ä}{{\"a}}1
  {ö}{{\"o}}1
  {~}{{\textasciitilde}}1
  {а}{{\selectfont\char224}}1
  {б}{{\selectfont\char225}}1
  {в}{{\selectfont\char226}}1
  {г}{{\selectfont\char227}}1
  {д}{{\selectfont\char228}}1
  {е}{{\selectfont\char229}}1
  {ё}{{\"e}}1
  {ж}{{\selectfont\char230}}1
  {з}{{\selectfont\char231}}1
  {и}{{\selectfont\char232}}1
  {й}{{\selectfont\char233}}1
  {к}{{\selectfont\char234}}1
  {л}{{\selectfont\char235}}1
  {м}{{\selectfont\char236}}1
  {н}{{\selectfont\char237}}1
  {о}{{\selectfont\char238}}1
  {п}{{\selectfont\char239}}1
  {р}{{\selectfont\char240}}1
  {с}{{\selectfont\char241}}1
  {т}{{\selectfont\char242}}1
  {у}{{\selectfont\char243}}1
  {ф}{{\selectfont\char244}}1
  {х}{{\selectfont\char245}}1
  {ц}{{\selectfont\char246}}1
  {ч}{{\selectfont\char247}}1
  {ш}{{\selectfont\char248}}1
  {щ}{{\selectfont\char249}}1
  {ъ}{{\selectfont\char250}}1
  {ы}{{\selectfont\char251}}1
  {ь}{{\selectfont\char252}}1
  {э}{{\selectfont\char253}}1
  {ю}{{\selectfont\char254}}1
  {я}{{\selectfont\char255}}1
  {А}{{\selectfont\char192}}1
  {Б}{{\selectfont\char193}}1
  {В}{{\selectfont\char194}}1
  {Г}{{\selectfont\char195}}1
  {Д}{{\selectfont\char196}}1
  {Е}{{\selectfont\char197}}1
  {Ё}{{\"E}}1
  {Ж}{{\selectfont\char198}}1
  {З}{{\selectfont\char199}}1
  {И}{{\selectfont\char200}}1
  {Й}{{\selectfont\char201}}1
  {К}{{\selectfont\char202}}1
  {Л}{{\selectfont\char203}}1
  {М}{{\selectfont\char204}}1
  {Н}{{\selectfont\char205}}1
  {О}{{\selectfont\char206}}1
  {П}{{\selectfont\char207}}1
  {Р}{{\selectfont\char208}}1
  {С}{{\selectfont\char209}}1
  {Т}{{\selectfont\char210}}1
  {У}{{\selectfont\char211}}1
  {Ф}{{\selectfont\char212}}1
  {Х}{{\selectfont\char213}}1
  {Ц}{{\selectfont\char214}}1
  {Ч}{{\selectfont\char215}}1
  {Ш}{{\selectfont\char216}}1
  {Щ}{{\selectfont\char217}}1
  {Ъ}{{\selectfont\char218}}1
  {Ы}{{\selectfont\char219}}1
  {Ь}{{\selectfont\char220}}1
  {Э}{{\selectfont\char221}}1
  {Ю}{{\selectfont\char222}}1
  {Я}{{\selectfont\char223}}1
  {і}{{\selectfont\char105}}1
  {ї}{{\selectfont\char168}}1
  {є}{{\selectfont\char185}}1
  {ґ}{{\selectfont\char160}}1
  {І}{{\selectfont\char73}}1
  {Ї}{{\selectfont\char136}}1
  {Є}{{\selectfont\char153}}1
  {Ґ}{{\selectfont\char128}}1
  {\{}{{{\color{brackets}\{}}}1 % Цвет скобок {
  {\}}{{{\color{brackets}\}}}}1 % Цвет скобок }
}
 
 % Для настройки заголовка кода
\usepackage{caption}
\DeclareCaptionFont{white}{\color{сaptiontext}}
\DeclareCaptionFormat{listing}{\parbox{\linewidth}{\colorbox{сaptionbk}{\parbox{\linewidth}{#1#2#3}}\vskip-4pt}}
\captionsetup[lstlisting]{format=listing,labelfont=white,textfont=white}
\renewcommand{\lstlistingname}{Код} % Переименование Listings в нужное именование структуры


%------------------------------------------------------------------------------

\author{Лукашенко Р.Б.}
\title{Программирование на С и С++}
\begin{document}
%\listoftodos
\maketitle

\tableofcontents{}

\chapter{Основные конструкции языка}
%############################################################
\section{Задание 1.1. Перевод дюймов в метрическую систему}
\subsection{Задание}

\hspace{\parindent}Задано целое число от 0 до 999. Определить сумму его цифр.

\subsection{Теоретические сведения}
\hspace{\parindent}Для реализации данной задачи были использованы стандартные функции ввода-вывода \texttt{scanf}, \texttt{printf}, \texttt{puts} из стандартной библиотеки языка \verb+С+, объявленные в заголовочном файле \textit{stdio.h}.

При помощи операторов ветвления \texttt{if-else} и \texttt{switch} реализовано интерактивное подменю для более удобного взаимодествия пользователя с программой.

\subsection{Проектирование}
\hspace{\parindent}В ходе проектирования было решено выделить 5 функций:
\begin{enumerate}
 	\item Вычисление суммы
 	
 	\verb+int c_calc_sum_of_digits(int);+
 	
	Параметром функции является число типа \verb+int+, которое вводит пользователь.
	Возвращается число типа \verb+int+, которое и является суммой разрядов.	 
		 
		 
	\item Меню с первоначальным пользовательским взаимодействием
	
	\verb+void c_sum_of_digits_ui();+
	
	Пользователю предлагается выбрать консольный ввод, вызов справки, возварт к главное меню или завершение программы.	
		 
		 
	\item Основное пользовательское взаимодействие
	
	\verb+void c_sum_of_digits_inp();+

	Пользователю предлагается ввести число, после чего вызывается функция для для решения задачи и вывода результата в консоль.
	
	
	\item Решение задачи и вывода результата в консоль
	
	\verb+void c_sum_of_digits_solution(int);+

	В качестве параметра передаётся число типа \verb+int+. Вызывается функция вычисления суммы. Результат выводится в консоль.	
	
	 
	\item Вспомогательная информация
	
	\verb+void c_sum_of_digits_help();+
	
	Вывод справки в консоль.
\end{enumerate}

\subsection{Описание тестового стенда и методики тестирования}
\hspace{\parindent}Среда разработки QtCreator 3.5.1, компилятор MinGW (x86 64 bit), операционная система Windows 10 64 bit.
В процессе выполнения задания производилось ручное тестирование.
Модульное тестирование реализовано при помощи фреймворка QtTest.

\subsection{Тестирование}
\begin{table}[h]
\caption{Тестирование вычисления суммы}
\label{sum_of_digits_test}
\begin{tabular}{|c|c|c|c|}
\hline 
Введённое число & Сумма разрядов & Тип теста & Результат \\ 
\hline 
1234 & 10 & Модульный & Успешно \\ 
\hline 
4321 & 10 & Ручной & Успешно \\ 
\hline 
\end{tabular} 
\end{table}
Все тесты пройдены успешно.
\subsection{Выводы}
\hspace{\parindent}При выполнении задания были закреплены навыки в работе с основными конструкциями языка \texttt{C} и получен опыт в организации многофайлового проекта и создании модульных тестов.

Данная задача также была переписана на языке \verb|C++| с использованием объектно-ориентироанного проектироания.
\newpage
\subsection{Листинги}
\verb+c_sum_of_digits.h+
\lstinputlisting[]
{../sources/SDP/cCalculations/c_sum_of_digits.h}
\verb+c_sum_of_digits.c+
\lstinputlisting[]
{../sources/SDP/cCalculations/c_sum_of_digits.c}
\verb+c_sum_of_digits_ui.h+
\lstinputlisting[]
{../sources/SDP/cMenu/c_sum_of_digits_ui.h}
\verb+c_sum_of_digits_ui.c+
\lstinputlisting[]
{../sources/SDP/cMenu/c_sum_of_digits_ui.c}




\newpage
%############################################################
\section{Задание 1.2. Поиск кратных}
\subsection{Задание}
\hspace{\parindent}Заданы три целых числа a, b и c. Найти среди них пары чисел, в которых одно число делится на 

другое.

\subsection{Теоритические сведения}
\hspace{\parindent}Были использованы стандартные функции ввода-вывода 
\texttt{scanf}, \texttt{printf}, \texttt{puts} из стандартной библиотеки языка \verb+С+, 
объявленные в заголовочном файле \textit{stdio.h}.

При помощи оператоов ветвления \texttt{if-else} и \texttt{switch} реализовано интерактивное подменю для более удобного взаимодествия пользователя с программой.


\subsection{Проектирование}
\hspace{\parindent}В ходе проектирования было решено выделить 5 функций:
\begin{enumerate}
	\item Поиск кратных
 	
 	\verb+int c_calc_multiples(int*);+
 	
	Параметром функции является массив из трёх чисел типа \verb+int+, которые вводит пользователь.
	Возвращается число типа \verb+int+, которое является количеством кратных чисел.
	Вывод производится прямо в этой функции.	 
		 
	\item Меню с начальным пользовательским взаимодействием
	
	\verb+void c_multiples_ui();+
	
	Пользователю предлагается выбрать консольный ввод, вызов справки, возварт к главное меню или завершение программы.
	
	\item Основное пользовательское взаимодействие
	
	\verb+void c_multiples_inp();+
	
	Пользователю предлагается последовательно ввести 3 числа, после чего вызывается функция для решения задачи и вывода результата в консоль.
	
	\item Решение задачи и вывода результата в консоль
	
	\verb+void c_multiples_solution(int*);+
	
	В качестве параметра передаётся массив из трёх чисел типа \verb+int+. Вызывается функция поиска. Результат выводится в консоль 

	\item Вспомогательная информация
	
	\verb+void c_multiples_help();+
	
	Вывод справки в консоль.
\end{enumerate}

\subsection{Описание тестового стенда и методики тестирования}
\hspace{\parindent}Среда разработки QtCreator 3.5.1, компилятор MinGW (x86 64 bit), операционная система Windows 10 64 bit.
В процессе выполнения задания производилось ручное тестирование.
Модульное тестирование реализовано при помощи фреймворка QtTest.

\subsection{Тестирование}
\begin{table}[h]
\caption{Тестирование нахождения кратных}
\label{multiples_test}
\begin{tabular}{| c c c | c | c | c |}
\hline 
m1 & m2 & m3 & Количество пар кратных & Тип теста & Результат \\ 
\hline 
2 & 4 & 8 & 3 & Модульный & Успешно \\ 
\hline 
8 & 4 & 2 & 3 & Ручной & Успешно \\ 
\hline 
\end{tabular} 
\end{table}
Все тесты пройдены успешно.

\subsection{Выводы}
\hspace{\parindent}При выполнении задания закреплены навыки в работе с основными конструкциями языка \verb+C+ и получен опыт в организации многофайлового проекта и создании модульных тестов.

Данная задача также была переписана на языке \verb|C++| с использованием объектно-ориентироанного проектироания.
\newpage
\subsection{Листинги}
\verb+c_multiples.h+
\lstinputlisting[]
{../sources/SDP/cCalculations/c_multiples.h}

\verb+c_multiples_ui.h+
\lstinputlisting[]
{../sources/SDP/cMenu/c_multiples_ui.h}
\verb+c_multiples_ui.c+
\lstinputlisting[]
{../sources/SDP/cMenu/c_multiples_ui.c}




\newpage
%############################################################
chapter{Циклы}
\section{Задание 2. Обратная запись числа}
\subsection{Задание}

\hspace{\parindent}Записать число наоборот.

\subsection{Теоретические сведения}
\hspace{\parindent}Были использованы стандартные функции ввода-вывода \texttt{scanf}, \texttt{printf}, \texttt{puts} из стандартной библиотеки языка \verb+С+, объявленные в заголовочном файле \textit{stdio.h}.

При помощи оператоов ветвления \texttt{if-else} и \texttt{switch} реализовано интерактивное подменю для более удобного взаимодествия пользователя с программой.

\subsection{Проектирование}
\hspace{\parindent}В ходе проектирования было решено выделить 5 функций:
\begin{enumerate}
 	\item Вычисление суммы
 	
 	\verb+int c_calc_reversed_num(int);+
 	
	Параметром функции является число типа \verb+int+, которое вводит пользователь.
	Возвращается число типа \verb+int+, которое и является обратной записью.	 
		 
		 
	\item Меню с первоначальным пользовательским взаимодействием
	
	\verb+void c_reversed_num_ui();+
	
	Пользователю предлагается выбрать консольный ввод, вызов справки, возварт к главное меню или завершение программы.	
		 
		 
	\item Основное пользовательское взаимодействие
	
	\verb+void c_reversed_num_inp();+

	Пользователю предлагается ввести число, после чего вызывается функция для решения задачи и вывода результата в консоль.
	
	
	\item Решение задачи и вывода результата в консоль
	
	\verb+void c_reversed_num_solution(int);+

	В качестве параметра передаётся число типа \verb+int+. Вызывается функция вычисления обратного числа. Результат выводится в консоль.	
	
	 
	\item Вспомогательная информация
	
	\verb+void c_reversed_num_help();+
	
	Вывод справки в консоль.
\end{enumerate}

\subsection{Описание тестового стенда и методики тестирования}
\hspace{\parindent}Среда разработки QtCreator 3.5.1, компилятор MinGW (x86 64 bit), операционная система Windows 10 64 bit.
В процессе выполнения задания производилось ручное тестирование.
Модульное тестирование реализовано при помощи фреймворка QtTest.

\subsection{Тестирование}
\begin{table}[h]
\caption{Тестирование обратного числа}
\label{sum_of_digits_test}
\begin{tabular}{|c|c|c|c|}
\hline 
Введённое число & Обратное число & Тип теста & Результат \\ 
\hline 
1234 & 4321 & Модульный & Успешно \\ 
\hline 
4321 & 1234 & Ручной & Успешно \\ 
\hline 
\end{tabular} 
\end{table}

Все тесты пройдены успешно.
\subsection{Выводы}
\hspace{\parindent}При выполнении задания были закреплены навыки в работе с основными конструкциями языка \texttt{C} и получен опыт в организации многофайлового проекта и создании модульных тестов.

Данная задача также была переписана на языке \verb|C++| с использованием объектно-ориентироанного проектироания.
\newpage
\subsection{Листинги}
\verb+c_reversed_num.h+
\lstinputlisting[]
{../sources/SDP/cCalculations/c_reversed_num.h}
\verb+c_reversed_num.c+
\lstinputlisting[]
{../sources/SDP/cCalculations/c_reversed_num.c}
\verb+c_reversed_num_ui.h+
\lstinputlisting[]
{../sources/SDP/cMenu/c_reversed_num_ui.h}
\verb+c_reversed_num_ui.c+
\lstinputlisting[]
{../sources/SDP/cMenu/c_reversed_num_ui.c}


\newpage
%############################################################
\chapter{Массивы}
\section{Задание 3. Поворот матрицы}
\subsection{Задание}

\hspace{\parindent}Содержимое квадратной матрицы A(n,n) повернуть на 90 градусов по часовой

стрелке вокруг центра матрицы.
\subsection{Теоретические сведения}
\hspace{\parindent}\hspace{\parindent}Были использованы стандартные функции ввода-вывода \texttt{scanf}, \texttt{printf}, \texttt{puts}, \texttt{fopen}, \texttt{fclose}, \texttt{fscanf}, \texttt{fprintf} из стандартной библиотеки языка \verb+С+, объявленные в заголовочном файле \textit{stdio.h}.

При помощи оператоов ветвления \texttt{if-else} и \texttt{switch} реализовано интерактивное подменю для более удобного взаимодествия пользователя с программой.

Так же использовались функции для работы с памятью \verb+malloc+ и \verb+free+, объявленные в заголовочном файле \textit{stdlib.h}.

\subsection{Проектирование}
\hspace{\parindent}В ходе проектирования было решено выделить 7 основных функций:
\begin{enumerate}
 	\item Поворот
 	
 	\verb+void c_calc_matrix_turn(int size_of_matrix, int **matrix)+
 	
	Параметрами функции являются: число типа \verb+int+, которое вводит пользователь(размер квадратной матрицы) и двумерный массив типа \verb+int**+(квадратная матрица).
		 
		 
	\item Меню с первоначальным пользовательским взаимодействием
	
	\verb+void c_matrix_turn_ui();+
	
	Пользователю предлагается выбрать консольный ввод, файловый ввод, вызов справки, возварт к главное меню или завершение программы. 	 
		 
	\item Консольное взаимодействие с пользователем
	
	\verb+void c_matrix_turn_cinp();+

	Пользователю предлагается ввести размер матрицы и саму матрицу, после чего вызывается функция для решения задачи и вывода результата в консоль.
	
	\item Файловое взаимодействие с пользователем
	
	\verb+void c_matrix_turn_finp();+

	Пользователю предлагается ввести имя файла, откуда программа должна получить исходные данные, после чего вызывается функция для решения задачи и вывода результата в файл.
	
	\item Решение задачи и вывода результата в консоль
	
	\verb+void c_matrix_turn_csolution(int, int**);+

	Параметрами функции являются: число типа \verb+int+, которое вводит пользователь(размер квадратной матрицы) и двумерный массив типа \verb+int**+(квадратная матрица). Вывод производится в консоль.
	
	\item Решение задачи и вывода результата в файл
	
	\verb+void c_matrix_turn_fsolution(int, int**);+

	Параметрами функции являются: число типа \verb+int+, которое вводит пользователь(размер квадратной матрицы) и двумерный массив типа \verb+int**+(квадратная матрица). Вывод производится в указанный файл результата.
	
	
	\item Вспомогательная информация
	
	\verb+void c_matrix_turn_help();+
	
	Вывод справки в консоль.
\end{enumerate}

\subsection{Описание тестового стенда и методики тестирования}
\hspace{\parindent}Среда разработки QtCreator 3.5.1, компилятор MinGW (x86 64 bit), операционная система Windows 10 64 bit.
В процессе выполнения задания производилось ручное тестирование.
Модульное тестирование реализовано при помощи фреймворка QtTest.

Все тесты пройдены успешно.
\subsection{Выводы}
\hspace{\parindent}При выполнении задания были закреплены навыки в работе с массивами и получен опыт в организации многофайлового проекта и создании модульных тестов.

Данная задача также была переписана на языке \verb|C++| с использованием объектно-ориентироанного проектироания.

\newpage
\subsection{Листинги}
\verb+c_matrix_turn.h+
\lstinputlisting[]
{../sources/SDP/cCalculations/c_matrix_turn.h}
\verb+c_matrix_turn.c+
\lstinputlisting[]
{../sources/SDP/cCalculations/c_matrix_turn.c}
\verb+c_matrix_turn_ui.h+
\lstinputlisting[]
{../sources/SDP/cMenu/c_matrix_turn_ui.h}
\verb+c_matrix_turn_ui.c+
\lstinputlisting[]
{../sources/SDP/cMenu/c_matrix_turn_ui.c}




\newpage
%############################################################
\chapter{Строки}
\section{Задание 4. Поиск ключей}
\subsection{Задание}

\hspace{\parindent}В заданном тексте подсчитать частоту использования каждого буквосочетания, слова и 

словосочетания из заданного списка..

\subsection{Теоретические сведения}
\hspace{\parindent}\hspace{\parindent}Были использованы стандартные функции ввода-вывода \texttt{scanf}, \texttt{printf}, \texttt{puts}, \texttt{fopen}, \texttt{fclose}, \texttt{fscanf}, \texttt{fprintf} из стандартной библиотеки языка \verb+С+, объявленные в заголовочном файле \textit{stdio.h}.

При помощи оператоов ветвления \texttt{if-else} и \texttt{switch} реализовано интерактивное подменю для более удобного взаимодествия пользователя с программой.

Для поиска ключей использовалась функция \verb+strstr+.

Так же использовались функции для работы с памятью \verb+malloc+ и \verb+free+, объявленные в заголовочном файле \textit{stdlib.h}.

\subsection{Проектирование}
\hspace{\parindent}В ходе проектирования было решено выделить 7 основных функций:
\begin{enumerate}
 	\item Поиск ключей
 	
 	\verb+void c_calc_keys_in_text(char**, char**, int, int, int*);+
 	
	Параметрами функции являются: 2 массива строк типа \verb+char**+(текст и ключевые слова), 2 числа типа \verb+int+(количество строк и ключей соответственно) и одномерный массив типа \verb+int*+(количество вхождений).
		 
		 
	\item Меню с первоначальным пользовательским взаимодействием
	
	\verb+void c_keys_in_text_ui();+
	
	Пользователю предлагается выбрать консольный ввод, файловый ввод, вызов справки, возварт к главное меню или завершение программы. 	 
		 
	\item Консольное взаимодействие с пользователем
	
	\verb+void c_keys_in_text_cinp();+

	Пользователю предлагается ввести текст и ключевые слова, после чего вызывается функция для решения задачи и вывода результата в консоль.
	
	\item Файловое взаимодействие с пользователем
	
	\verb+void c_keys_in_text_finp();+

	Пользователю предлагается ввести имя файла, откуда программа должна получить исходные данные, после чего вызывается функция для решения задачи и вывода результата в файл.
	
	\item Решение задачи и вывода результата в консоль
	
	\verb+void c_keys_in_text_csolution(char**, char**, int, int);+

	Параметрами функции являются: 2 массива строк типа \verb+char**+(текст и ключевые слова), 2 числа типа \verb+int+(количество строк и ключей соответственно). Вывод производится в консоль.
	
	\item Решение задачи и вывода результата в файл
	
	\verb+void c_keys_in_text_fsolution(char**, char**, int, int);+

	Параметрами функции являются: 2 массива строк типа \verb+char**+(текст и ключевые слова), 2 числа типа \verb+int+(количество строк и ключей соответственно). Вывод производится в указанный файл результата.
	
	
	\item Вспомогательная информация
	
	\verb+void c_keys_in_text_help();+
	
	Вывод справки в консоль.
\end{enumerate}

\subsection{Описание тестового стенда и методики тестирования}
\hspace{\parindent}Среда разработки QtCreator 3.5.1, компилятор MinGW (x86 64 bit), операционная система Windows 10 64 bit.
В процессе выполнения задания производилось ручное тестирование.
Модульное тестирование реализовано при помощи фреймворка QtTest.

Все тесты пройдены успешно.
\subsection{Выводы}
\hspace{\parindent}При выполнении задания были закреплены навыки в работе со строками и получен опыт в организации многофайлового проекта и создании модульных тестов.

Данная задача также была переписана на языке \verb|C++| с использованием объектно-ориентироанного проектироания.

\newpage
\subsection{Листинги}
\verb+c_keys_in_text.h+
\lstinputlisting[]
{../sources/SDP/cCalculations/c_keys_in_text.h}
\verb+c_keys_in_text.c+
\lstinputlisting[]
{../sources/SDP/cCalculations/c_keys_in_text.c}
\verb+c_keys_in_text_ui.h+
\lstinputlisting[]
{../sources/SDP/cMenu/c_keys_in_text_ui.h}
\verb+c_keys_in_text_ui.c+
\lstinputlisting[]
{../sources/SDP/cMenu/c_keys_in_text_ui.c}





\newpage
%############################################################
\chapter{Инкапсуляция}
\section{Задание 5. БОЛЬШОЕ ЦЕЛОЕ ЧИСЛО}
\subsection{Задание}

\hspace{\parindent}Реализовать класс БОЛЬШОЕ ЦЕЛОЕ ЧИСЛО (максимальное значение неограничено). Требуемые 

методы: конструктор, деструктор, копирование, сложение, вычитание, умножение, преобразование к 

типу int. 

\subsection{Теоретические сведения}
\hspace{\parindent}Для реализации данной задачи были использованы библиотечные потоки ввода-вывода \verb+STL+, такие как \verb+cin+ и \verb+cout+, объявленные в заголовочном файле \textit{iostream}.

При помощи цикла \verb+for+ и происходит иттерирование по массиву. Работа с памятью происходит при помощи операторов \verb+new+ и \verb+delete+.

Для создания данного класса использован \verb+vector+, т.к. количество разрядов бцч не ограничено. Все действия выполнялись порахрядно, наподобии вычисления в столбик.

\subsection{Проектирование}
\hspace{\parindent}В ходе проектирования было решено выделить класс \verb+cpp_int_unlim+ 
для представления бцч и класс \verb+cpp_int_unlim_inp+, занимающийся вводом и выводом бцч в консоль.

Класс \verb+cpp_int_unlim+ было решено спроектировать следующим образом: 
\begin{enumerate}
 	\item Поле, содержащие бцч и результаты вычислений
 	
 	\verb+vector<int> num1;+
    \verb+vector<int> num2;+
    \verb+vector<int> min_num;+
    \verb+vector<int> max_num;+
    \verb+vector<int> sum;+
    \verb+vector<int> subt;+
    \verb+vector<int> mult;+
    \verb+int num1_to_i;+
    \verb+int num2_to_i;+
		 
	\item Поле, содержащее промежуточные вычисления
	
	\verb+int size_of_num1;+
    \verb+int size_of_num2;+
    \verb+int size_of_min;+
    \verb+int size_of_max;+
    \verb+int size_of_result;+
    
    \item Поле, содержащее вспомогательные переменные
	
	\verb+int trigger;+
    \verb+int test;+
		 
	\item Конструкторы
	
	\verb+cpp_int_unlim();+

	Конструктор по умолчанию.
	
	\verb+Array(Array &);+
	
	Конструктор. Пользователь вводит два бцч.
	
	\verb+cpp_int_unlim(vector<int>, vector<int>);+

	Конструктор копирования инициализирует создаваемый объект с помощью другого объекта этого класса.
	
	\verb+cpp_int_unlim(const cpp_int_unlim& obj);+
	
	\item Отдельный метод для ввода бцч
	
	\verb+void enter_numbers(vector<int>, vector<int>);+
	
	\item Метод, для промежуточных вычислений(размер каждого числа, какое больше, а какое меньше и т.д.).
	
	\verb+void basic_calculations();+
	
	Вызов обязателен для *, -, +.
	
	\item Метод для сложения
	
	\verb+void sum_of_iu();+
	
	\item Метод для вычитания
	
	\verb+void subtraction_of_iu();+
	
	\item Метод для умножения
	
	\verb+void multiplication_of_iu();+
	
	\item Метод для перевода в int
	
	\verb+void iu_to_i();+
	
	\item Методы для вывода
	
	\verb+void get_...();+
	
	\item Деструктор	
	
	\verb+~cpp_int_unlim();+
	
	В деструкторе происходит удалание выделенной памяти под хранение векторов с помощью \verb+.clear();+.
	
\end{enumerate}

Класс \verb+cpp_int_unlim_inp+ было решено спроектировать следующим образом: 
\begin{enumerate}
 	\item Поле, содержащее объект класса \verb+Array+
 	
 	\verb+cpp_int_unlim iu;+
 	
	\item Конструктор
	
	\verb+cpp_int_unlim_inp();+

	\item Метод, осуществляющий ввод бцч
	
	\verb+void cpp_iu_inp();+
	
	\item Деструктор	
	
	\verb+~~cpp_int_unlim_inp();+
	
\end{enumerate}

\subsection{Тестирование}
\begin{table}[h]
\caption{Тестирование обратного числа}
\label{sum_of_digits_test}
\begin{tabular}{|c|c|c|c|c|c|}
\hline 
Действие & Введённое число1 & Введённое число2 & Результат & Тип теста & Результат \\ 
\hline 
Сложение & 654321 & 123456 & 777777 & Модульный & Успешно \\ 
\hline 
Вычитание & 654321 & 123456 & 530865 & Модульный & Успешно \\ 
\hline 
Умножение & 4321 & 1234 & 5332114 & Модульный & Успешно \\ 
\hline 
\end{tabular} 
\end{table}
Все тесты пройдены успешно.
\subsection{Выводы}
\hspace{\parindent}При выполнении задания были усвоены навыки проектирования и создания классов и получен опыт в организации многофайлового проекта.

\newpage
\subsection{Листинги}
\verb+cpp_int_unlim.h+
\lstinputlisting[]
{../sources/SDP/cppCalculations/cpp_int_unlim.h}
\verb+cpp_int_unlim.cpp+
\lstinputlisting[]
{../sources/SDP/cppCalculations/cpp_int_unlim.cpp}
\verb+cpp_int_unlim_inp.h+
\lstinputlisting[]
{../sources/SDP/cppMenu/cpp_int_unlim_inp.h}
\verb+cpp_int_unlim_inp.cpp+
\lstinputlisting[]
{../sources/SDP/cppMenu/cpp_int_unlim_inp.cpp}

\chapter*{Приложения}
\subsection*{Листинги модульных тестов к заданиям с 1 по 4 включительно}
\lstinputlisting[label=unit_tests_1_4]
{../sources/SDP/cTests/tst_cteststest.cpp}
\newpage

\subsection*{Листинги класса $cpp_sum_of_digits$}
\lstinputlisting[label=class_cpp_sum_of_digits]
{../sources/SDP/cppCalculations/cpp_sum_of_digits.h}
\lstinputlisting[label=class_cpp_sum_of_digits]
{../sources/SDP/cppCalculations/cpp_sum_of_digits.cpp}
\newpage

\subsection*{Листинги класса $cpp_multiples$}
\lstinputlisting[]
{../sources/SDP/cppCalculations/cpp_multiples.h}

\newpage

\subsection*{Листинги класса $cpp_reversed_num$}
\lstinputlisting[label=class_cpp_reversed_num]
{../sources/SDP/cppCalculations/cpp_reversed_num.h}
\lstinputlisting[label=class_cpp_reversed_num]
{../sources/SDP/cppCalculations/cpp_reversed_num.cpp}
\newpage

\subsection*{Листинги класса $cpp_matrix_turn$}
\lstinputlisting[label=class_cpp_matrix_turn]
{../sources/SDP/cppCalculations/cpp_matrix_turn.h}
\lstinputlisting[label=class_cpp_matrix_turn]
{../sources/SDP/cppCalculations/cpp_matrix_turn.cpp}
\newpage

\subsection*{Листинги класса $cpp_keys_in_text$}
\lstinputlisting[label=class_cpp_keys_in_text]
{../sources/SDP/cppCalculations/cpp_keys_in_text.h}
\lstinputlisting[label=class_cpp_keys_in_text]
{../sources/SDP/cppCalculations/cpp_keys_in_text.cpp}
\newpage

\subsection*{Листинги модульных тестов к заданиям с 1 по 4 включительно на языке C++}
\lstinputlisting[label=arrayappdemo]
{../sources/SDP/cpptests/tst_cppteststest.cpp}
\newpage

\end{document}